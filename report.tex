%!TEX program = xelatex
\documentclass[11pt, a4paper]{article}

\usepackage{amsmath}
\usepackage{listings}
\usepackage{amssymb}

% fonts
\usepackage{xeCJK}
\setCJKmainfont[BoldFont=SimHei]{SimSun}
\setCJKfamilyfont{hei}{SimHei}
\setCJKfamilyfont{kai}{KaiTi}
\setCJKfamilyfont{fang}{FangSong}
\newcommand{\hei}{\CJKfamily{hei}}
\newcommand{\kai}{\CJKfamily{kai}}
\newcommand{\fang}{\CJKfamily{fang}}

% style
\usepackage[top=2.54cm, bottom=2.54cm, left=3.18cm, right=3.18cm]{geometry}
\linespread{1.5}
\usepackage{indentfirst}
\parindent 2em
\punctstyle{quanjiao}
\renewcommand{\today}{\number\year 年 \number\month 月 \number\day 日}

% figures and tables
\usepackage{graphicx}
\usepackage[font={bf, footnotesize}, textfont=md]{caption}
\makeatletter
    \newcommand\fcaption{\def\@captype{figure}\caption}
    \newcommand\tcaption{\def\@captype{table}\caption}
\makeatother
\usepackage{booktabs}
\renewcommand\figurename{图}
\renewcommand\tablename{表}
\newcommand{\fref}[1]{\textbf{图 \ref{#1}}}
\newcommand{\tref}[1]{\textbf{表 \ref{#1}}}
\newcommand{\tabincell}[2]{\begin{tabular}{@{}#1@{}}#2\end{tabular}} % multiply lines in one grid
\usepackage{longtable} % long table

\usepackage{listings}
\lstset{basicstyle=\ttfamily}

\usepackage{xcolor}
\renewcommand{\r}{\color{red}}
\usepackage{tabulary}
% start of document
\title{\textbf{搜索引擎大作业报告}}
\author{
    \kai 钱迪晨 \quad 计35 \quad 2013011402 \\
    \kai 温和 \quad 计35 \quad 20130114XX
}
\date{\kai\today}

% -----------------start here------------------%
\begin{document}
\lstset{                        %Settings for listings package.
  language=[ANSI]{C++},
  % backgroundcolor=\color{lightgray},
  basicstyle=\footnotesize,
  breakatwhitespace=false,
  breaklines=true,
  captionpos=b,
  commentstyle=\color{olive},
  directivestyle=\color{blue},
  extendedchars=false,
  % frame=single,%shadowbox
  framerule=0pt,
  keywordstyle=\color{blue}\bfseries,
  morekeywords={*,define,*,include...},
  numbersep=5pt,
  rulesepcolor=\color{red!20!green!20!blue!20},
  showspaces=false,
  showstringspaces=false,
  showtabs=false,
  stepnumber=2,
  stringstyle=\color{purple},
  tabsize=4,
  title=\lstname
}

\maketitle

\section{爬虫}

\section{文本解析}

\section{lucene}

% \begin{split}
% S &= \frac{1}{(1-0.5) + \frac{0.5}{10}}\\
% S &= \frac{1}{0.5 + 0.05} \\
% S &= 1.818
% \end{split}
% \]
% \begin{center}
%     \includegraphics[height=10cm]{result/9_epoch_1.jpg}
%     \fcaption{第1轮迭代训练的结果}\label{1}
% \end{center}

% \begin{enumerate}
%     \item 随着实验的进行,图片的确变得更加的准确,而不仅仅是肉眼上的复原效果更加好了,但是训练的次数过多也会使得画质变得更差。
%     \item 10次迭代的时候,答案是最优的。
%     \item 绿色通道随着迭代次数上升最准确,但是红色和蓝色通道就有一点不正确。
% \end{enumerate}


% \begin{center}
%     \tcaption{内存地址空间映射}\label{table:mem_addr}
%     \begin{longtable}{ll}
%         \toprule
%         逻辑地址 & 映射到的物理地址或设备 \\
%         \midrule
%         0x0000 $\thicksim$ 0x7FFF & RAM2 0x0000 $\thicksim$ 0x7FFF \\
%         0x8000 $\thicksim$ 0xBEFF & RAM1 0x0000 $\thicksim$ 0x3EFF \\
%         \bottomrule
%     \end{longtable}
% \end{center}

\end{document}
% -----------------end------------------%
