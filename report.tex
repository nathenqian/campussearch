%!TEX program = xelatex
\documentclass[11pt, a4paper]{article}

\usepackage{amsmath}
\usepackage{listings}
\usepackage{amssymb}

% fonts
\usepackage{xeCJK}
\setCJKmainfont[BoldFont=SimHei]{SimSun}
\setCJKfamilyfont{hei}{SimHei}
\setCJKfamilyfont{kai}{KaiTi}
\setCJKfamilyfont{fang}{FangSong}
\newcommand{\hei}{\CJKfamily{hei}}
\newcommand{\kai}{\CJKfamily{kai}}
\newcommand{\fang}{\CJKfamily{fang}}

% style
\usepackage[top=2.54cm, bottom=2.54cm, left=3.18cm, right=3.18cm]{geometry}
\linespread{1.5}
\usepackage{indentfirst}
\parindent 2em
\punctstyle{quanjiao}
\renewcommand{\today}{\number\year 年 \number\month 月 \number\day 日}

% figures and tables
\usepackage{graphicx}
\usepackage[font={bf, footnotesize}, textfont=md]{caption}
\makeatletter
    \newcommand\fcaption{\def\@captype{figure}\caption}
    \newcommand\tcaption{\def\@captype{table}\caption}
\makeatother
\usepackage{booktabs}
\renewcommand\figurename{图}
\renewcommand\tablename{表}
\newcommand{\fref}[1]{\textbf{图 \ref{#1}}}
\newcommand{\tref}[1]{\textbf{表 \ref{#1}}}
\newcommand{\tabincell}[2]{\begin{tabular}{@{}#1@{}}#2\end{tabular}} % multiply lines in one grid
\usepackage{longtable} % long table

\usepackage{listings}
\lstset{basicstyle=\ttfamily}

\usepackage{xcolor}
\renewcommand{\r}{\color{red}}
\usepackage{tabulary}
% start of document
\title{\textbf{搜索引擎大作业报告}}
\author{
    \kai 钱迪晨 \quad 计35 \quad 2013011402 \\
    \kai 温和 \quad 计35 \quad 2013011407
}
\date{\kai\today}

% -----------------start here------------------%
\begin{document}
\lstset{                        %Settings for listings package.
  language=[ANSI]{C++},
  % backgroundcolor=\color{lightgray},
  basicstyle=\footnotesize,
  breakatwhitespace=false,
  breaklines=true,
  captionpos=b,
  commentstyle=\color{olive},
  directivestyle=\color{blue},
  extendedchars=false,
  % frame=single,%shadowbox
  framerule=0pt,
  keywordstyle=\color{blue}\bfseries,
  morekeywords={*,define,*,include...},
  numbersep=5pt,
  rulesepcolor=\color{red!20!green!20!blue!20},
  showspaces=false,
  showstringspaces=false,
  showtabs=false,
  stepnumber=2,
  stringstyle=\color{purple},
  tabsize=4,
  title=\lstname
}

\maketitle

\section{爬虫}
爬虫使用的是python自己写的爬虫。我们使用eventlet协程来控制IO,而不使用多线程进行IO控制,大大降低了CPU的负荷。实际测试中,可以同时支持300个HTTP请求的IO,速度非常快,如果是多线程的话只能达到大概50个。当然最后爬数据的时候,只开了10个协程。但是速度非常快,大概8个小时就爬了20万网页。

数据存储在本地,每1万个数据存储一个文件夹。每个数据有原始数据以及json格式的描述文件,包含一些信息。

爬虫也做了中断保存的功能,可以随时中断,然后继续上一次的任务队列。同时爬虫也做了简单的命令行指令,可以随时加入种子网页,以及保存退出。

\section{文件解析}
文本解析使用Apache的Tika(对于office文档和pdf)和python中的BeautifulSoup4(对于html)完成。对于office文档,直接使用Tika提取文件中的文本和作者等信息;对于html,使用bs4提取body中的文本。对于爬虫得到的20万个结果,共处理了10小时左右。

对于html文件,我们还提取了一张body中的图片(去除明显的banner、logo等图片),作为显示搜索结果时的预览图。

\section{lucene}
lucene使用了最新的6.0.分词器使用的是smartCNAnalayzer,效果很不错。原来作业里面的lucene版本太老旧。

建立索引的时候,将pagerank的值乘到每个document的每个field里面去。(lucene6.0不支持document boost)。同时在建立的索引的时候,将一些信息也保存进去,比如这个document的原网址,网址的图片,这些用来渲染的必须信息。由于pagerank在量级上差距过大,我们取系数为\(1.5*(log_2(pr)+22)\)。

搜索的时候使用的query是MultiFieldQueryParser用来构建询问。关于评分函数,我们使用了标题,文本和作者,权重分别为1,0.01和0.5,对于html还有额外的h1、h2、h3和a,权重分别为0.05,0.03和0.02,。

\section{前端}
网站前端我们使用的是semantic2.1,以及jquery。

搜索初始页只有一个标题和搜索框,非常简洁。

\begin{center}
    \includegraphics[width=5in]{index}
\end{center}

页面最上面是一个搜索栏,显示当前的搜索关键词。
结果页面分成两部分,左半边是搜索的结果以及页面的切换。右半边是页面的浏览功能,这个功能如果是有上下滑动的触摸板的话效果非常好。
效果如下图:

\begin{center}
    \includegraphics[width=5in]{result}
\end{center}

\section{代码详细}
在src文件夹中包含三部分。

crawler是由python写的爬虫部分。

script是由python写的进行数据处理的部分。

webserver是使用lucene6.0,java写的搜索器。

\end{document}
% -----------------end------------------%
